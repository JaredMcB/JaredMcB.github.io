\documentclass[12pt]{amsart}
% The srcltx package helps with inverse dvi search
%\usepackage{srcltx}

% This package has an easy way to set the margins
\usepackage[paper=letterpaper]{geometry}
\geometry{top=1in,left=1in,right=1in,bottom=1in,headheight=14pt}

% These packages include nice commands from AMS-LaTeX
\usepackage{algorithm}
\usepackage{algorithmic}
\usepackage{amssymb,amsmath,amsthm}
\usepackage{bm}
\usepackage{booktabs}
\usepackage{cancel}
\usepackage{caption}
\usepackage{changepage}
\usepackage{color}
\usepackage{comment}
\usepackage{empheq}
\usepackage[shortlabels]{enumitem}
\usepackage[mathscr]{euscript}
\usepackage{float}
\usepackage{graphicx}
\usepackage{hyperref}
%\usepackage[linktoc=none]{hyperref}
\usepackage{listings}
\usepackage{lscape}
\usepackage{multimedia}
\usepackage{setspace}
\usepackage{subfig}
\usepackage{tabularx}
\usepackage{textcomp}
\usepackage{tikz}
\usetikzlibrary{patterns}
\usetikzlibrary{scopes}
\usetikzlibrary{decorations.pathmorphing}
\usetikzlibrary{shapes,backgrounds, arrows}
\usepackage{xcolor}

%This package helps make nice headers
\usepackage{fancyhdr}
\pagestyle{fancy}
\lhead{\textsf{Note}}
\chead{\textsf{Jared McBride}}
\rhead{\textsf{\today}}
\renewcommand{\headrulewidth}{1pt}

% Make the space between lines slightly more
% generous than normal single spacing, but compensate
% so that the spacing between rows of matrices still
% looks normal.  Note that 1.1=1/.9090909...
\renewcommand{\baselinestretch}{1.1}
\renewcommand{\arraystretch}{1.1}

\renewcommand{\and}{\qquad\text{and}\qquad}
\newcommand{\ds}{\displaystyle}

\DeclareMathOperator*{\argmax}{argmax}
\DeclareMathOperator*{\argmin}{argmin}

\newcommand\Fontvi{\fontsize{8}{7.2}\selectfont}
\newcommand{\vs}{\vspace{2 cm}}
\newcommand{\E}{\mathbb{E}}
\newcommand{\p}{\mathbb{P}}
\newcommand{\T}{\mathbb{T}}
\newcommand{\R}{\mathbb{R}}
\newcommand{\C}{\mathbb{C}}
\newcommand{\Q}{\mathbb{Q}}
\newcommand{\Z}{\mathbb{Z}}
\newcommand{\N}{\mathbb{N}}
\newcommand{\F}{\mathscr{F}}
\newcommand{\scrE}{\mathscr{E}}
\newcommand{\spa}{\text{span}}
\newcommand{\rref}{\mathrm{rref}}

\newcommand{\X}{\mathfrak{X}}
\newcommand{\ssX}{\bm{\textsf{X}}}
\newcommand{\ssx}{\bm{\textsf{x}}}
\newcommand{\fS}{\mathfrak{S}}
\newcommand{\fC}{\mathfrak{C}}

\renewcommand{\b}[1]{{\color{blue} #1}}
\usepackage[T1]{fontenc}
\usepackage{beramono}
\usepackage{listings}

\renewcommand{\and}{\qquad\text{and}\qquad }

\renewcommand{\P}{\mathbb{P}}
\newcommand{\cov}{\mathrm{cov}}
\newcommand{\var}{\mathrm{var}}
\newcommand{\abox}[1]
{
	\begin{center}
		\fbox{$\displaystyle #1$}
	\end{center}
}
%%
%% Julia definition (c) 2014 Jubobs
%%
\lstdefinelanguage{Julia}%
{morekeywords={abstract,break,case,catch,const,continue,do,else,elseif,end,export,false,for,function,immutable,import,importall,if,in,macro,module,otherwise,quote,return,switch,true,try,type,typealias,using,while},%
	sensitive=true,%
	alsoother={$},%
	morecomment=[l]\#,%
	morecomment=[n]{\#=}{=\#},%
	morestring=[s]{"}{"},%
	morestring=[m]{'}{'},%
}[keywords,comments,strings]%

\lstset{%
	language         = Julia,
	basicstyle       = \ttfamily,
	keywordstyle     = \bfseries\color{blue},
	stringstyle      = \color{magenta},
	commentstyle     = \color{olive},
	showstringspaces = false,
}


\newtheorem{theorem}{Theorem}
\newtheorem{lemma}{Lemma}

\title{Numerical Spectral Factorization by Kalman Recursion}
\author{Jared A. McBride}
\date{\today}

\doublespacing

\begin{document}
	
\begin{abstract}
	In this note I describe a method presented by Kailath et al. in \cite{sayed2001} and \cite{kailath2000} for finding a spectral factor of the $z$-spectrum of a (wide-sense stationary) stochastic process under certain conditions. The method employs a version of Kalman filtering. As such I begin with a summary of relevant results from Kalman filtering. With those results in hand I present the method for finding spectral factors and conclude with examples.
\end{abstract}

\maketitle
\tableofcontents

\section{Introduction}

Given a wide-sense stationary (WSS) stochastic process $X = \big(X_n, n \in \Z\big)$ it's $z$-spectrum $S_X(z)$ is defined by\footnote{
	A note on notation: the superscript $*$ denotes the complex conjugate transpose (Hermitian transpose) operator for matrices and just the complex conjugate for real or complex numbers. The superscripts $-*$ together is an abbreviation for the complex conjugate of the multiplicative inverse, i.e. $$z^{-*} = \frac{1}{z^*}.$$
	} 
$$S_X(z) = \sum_{n=-\infty}^\infty C_X[n]z^{-n},$$
where $C_X[n] = \E X_nX_0^*$ is the autocovariance sequence of the process $X$. This function is real-valued on the unit circle since, $C_X[-n] = C_X^*[n]$, and by the Wiener-Khinchin theorem, it is, in fact, nonnegative definite on the unit circle\footnote{
	Though what follows holds for scalar processes it also holds for vector valued processes. As such, I assume the vector case. Meaning, the covariance sequence is a sequence of square $d\times d$-matrices, where $d$ is the dimension of $X_0$.}
Spectral factorization refers to a particular factorization in which 
$$S_X(z) = L(z)L^*(z^{-*})$$
where $L(z)$ is \emph{minimum phase}, meaning both $L(z)$ and $L^{-1}(z) = 1/L(z)$ are analytic on and outside the unit circle. The function $L$ will be referred to as a spectral factor of $S_X$. This factorization is not unique since for any spectral factor of $S_X$ right-multiplication by a unitary matrix produces another spectral factor of $S_X$. 

In this note I describe a method presented by Kailath et al. in \cite{sayed2001} and \cite[p.~336]{kailath2000} for finding a spectral factor under certain conditions. The method employs a version of Kalman filtering, more specifically the Kalman recursion for finding the innovations process for a process with a finite state-space model. The innovations for such a process will be defined below. Section \ref{sec: Kalman} provides a summary of relevant results from Kalman filtering, together with some intuition relevant to the present context. With those results in hand, Section \ref{sec: Factor} presents the method for finding spectral factors. Then, in Section \ref{sec: Example} are examples.  

\section{Kalman Filtering}
\label{sec: Kalman}

The purpose of this section is to provide information from Kalman filtering theory that is pertinent to the development of the factorization algorithm to be discussed. To that end, the \textit{summum bonum} of the section is the computation of what I refer to as a \emph{weak} modeling filter for a given process. The associated \emph{weak} whitening filter also bears significance in this work and will be briefly treated.   

Suppose we have a (discrete-time) stochastic process\footnote{
	From here till Section \ref{sec: Kalman stationary} $X$ need not be WSS.
}
$X = \big(X_i,\; i =  0,1,2,\dots\big)$ with a finite state-space representation of the following form:
\begin{subequations}
	\label{equ: ss}
	\begin{empheq}[left=\empheqlbrace]{align}
		\label{equ: ss State}\theta_{i+1} &= F_i\theta_i + G_iu_i \\
		\label{equ: ss Obs}X_i &= H_i\theta_i + v_i
	\end{empheq}
	where $F_i \in C^{m\times m},$ $G_i \in C^{m\times p},$ and $H_i \in C^{d\times m}$ are known matrices, and $u=(u_i)$, $v=(v_i)$, and $\theta_0$ are random variables with the following property\footnote{
		Throughout the note $$\delta_{ij} = \begin{cases} 1 &\text{if }i = j\\ 0 &\text{if }i \ne j.\end{cases}$$
		}
	\begin{equation}
	\E\begin{pmatrix} \theta_0 \\u_i \\ v_i \end{pmatrix}
	\begin{pmatrix} \theta_0 \\u_j \\ v_j\\1 \end{pmatrix}^*
	=\begin{pmatrix}
	\Pi_0 & 0                & 0              & 0 \\
	0     & Q_i\delta_{ij}   & S_i\delta_{ij} & 0 \\
	0     & S^*_i\delta_{ij} & R_i\delta_{ij} & 0
	\end{pmatrix}
	\label{equ: cond}
	\end{equation}
\end{subequations}

In the context here, I think of $X$ as a given series of\textit{ observations}, they are directly measurable, and $\theta = \big(\theta_i,\; i = 0,1,2,\dots\big)$ is the \textit{state} of the process (usually not directly measurable). Both of these series are sequences of random variables, along with $u$ and $v$. The matrices $F_i$, $G_i$, $H_i$, $\Pi_0$, $Q_i$, $S_i$ and $R_i$ are deterministic for all $i$. 

\subsection{A few words about the mathematical context}
In the sequel there will be much disccusion on  linear least-mean-square estimators
A nice thing about least-mean-square estimators is an orthogonality principle, namely the estimator error is orthogonal to the predictor. What follows also affords such a principle. However, there is no Hilbert space that comfortable houses this theory. So, I will build up the context here. 

The context requires the iterations of vectors of more than one size, since the observations and the state need not be the same dimension. Since it is possible to write solutions the the problems I consider in terms of covariance information a natural ``inner product'' may be defined by
\begin{equation}
\langle x,y \rangle = \E xy^*\qquad \text{for all }x\in \C^d,y\in \C^m, \qquad d,m \in \N,
\label{def: inner p}
\end{equation}
So, that $\langle x,x \rangle \in \C^{d\times d}$ is the covariance of $x$ and $\langle x,y \rangle \in \C^{d\times m}$ is the cross covariance of $x$ and $y$. We use the notation $x \perp y$ if $x$ and $y$ are uncorrelated, meaning $\langle x,x \rangle = 0 \in \C^{d\times m}$.  

Because this inner product\footnote{
	If its arguments are restricted to the same space, It can be shown that $\langle \cdot, \cdot\rangle$ satisfies the axioms of an inner product, provided positivity is replaced with positive defitiveness.}
is matrix-valued, rather than adding structure to a linear space with a scalar field the underlining space must be a module of vectors, with a ring of (square) matrix-valued scalars.The natural choice of scalars are matrices, of the appropriate size. This is an important point and leads me to another definition.
Let $$\text{span}_d\{y_i,\;i=0,1,\dots,N\} := \left\{\sum_{i=0}^N A_iy_i ~\bigg|~ y_i\in \C^m,A_i \in \C^{d\times m},\quad i=0,1,\dots,N\right\}$$
I would like us to think of the span as possiblily set in a different space from the arguments.

There is a norm 
\begin{equation}
\|x\|=\sqrt{\langle x,x \rangle}
\label{def: norm}
\end{equation} associated with this inner product, defined in the usual way, using the unique (Hermitian) positive-semidefinite square root of the Hermitian positive-semidefinite matrix $\langle x,x \rangle$.    

\subsection{the Kalman filter}
Now, the Kalman filter allows us to find, for some $i>0$, the linear least-mean-squares estimator of the state $\theta_{i}$ given all preceding observations $X_j$ for $j = 0,1,2,\dots, i-1$. 
This quantity I will denote by $\hat{\theta}_{i|i-1}$. Since in this note I will only be concerned with linear least-mean-squares estimators given the $X_i$'s the subscript $\cdot_{|i}$ will denote linear estimators with respect to and conditioned on all $X_j$ for $j \le i$. 
Consistent with this convention, when I write $\hat{X}_{i|i-1}$ I mean the linear least-mean-squares estimator of $X_i$ given $X_j$ for $j < i$. 

For the stake of rigor and completeness I have provided lemma which describes the orthogonality principle in this context. For the sake of brevity I have placed the proof in Appendix \ref{app: Ortho proof}.

\begin{lemma}
	\label{lem: otho p}
	Given the inner product defined under (\ref{def: inner p}) and the norm induced by it, under (\ref{def: norm}), let $x\in C^d$ and $y_i \in \C^m$ for $i=0,\dots,N$ be random variables. If $\hat x$ is the linear least-mean-squares estimator of $x$ given $y$, meaning 
	$$\E(x-\hat x)(x-\hat x)^* = \|x-\hat x\|^2 = \inf_{x'\in \text{\emph{span}}_d\{y_i,\;\forall i\}} \|x-x'\|^2$$
	then $(x-\hat x) \perp y_i$ for $i=0,1,\dots,N$. 
\end{lemma}

We seek a least-mean-square estimator of $\theta_{i+1}$ redistricted to a linear combination of observations, $\{X_j,\;j < i\}$. In this context we seek $\hat\theta_{i+1|i}$ such that,
\begin{align}
	\hat\theta_{i+1|i} &\in\mathcal{L}_i =: \text{span}\{X_i,\;i=0,1,\dots,N\}\\
	\label{equ: cond proj}\hat\theta_{i+1|i} &= \displaystyle\argmin_{\hat\theta \in \mathcal{L}_i}\|\theta_{i+1} - \hat\theta\|^2
\end{align}
Borrowing from Hilbert space theory, we consider building an orthogonal basis for $\mathcal{L}_i$, $e = \{e_j,\;j = 1,2,\dots,i\}$ and write 
\begin{equation}
\hat{\theta}_{i+1|i} = \sum_{j=0}^i \langle \theta_{i+1}, e_j \rangle R^{-1}_{e,j} e_j\label{equ: projection}
\end{equation}
where $R_{e,i} = \langle e_i, e_i \rangle$. We are not in a Hilbert space, so the claim that the estimator under (\ref{equ: projection}) satisfies (\ref{equ: cond proj}) will need to be verified. But first, let us build this orthogonal set in a way similar to the Grahm-Schmidt process, but without normalizing. Put
$$e_i = X_i - \hat{X}_{i|i-1}\quad \text{for }i>0\qquad\text{and}\qquad e_0 = X_0$$
The $\{e_i\}$ have the property $\mathcal{L}_i = \text{span}\{e_j,~j\le i\}=\mathcal{L}_i$ for any $i\ge 0$, and $\langle e_i,e_j \rangle = R_{e,i}\delta_{ij}$ by Lemma \ref{lem: otho p}.




\vspace{3cm}

This stochastic process we will refer to as the \textit{innovations} associated with $X$. Observe that because they are defined as the residual of least squares estimates over a progressively increasing space  the innovations are an orthogonal set. And we having $\langle e_i,e_j \rangle = I\delta_{ij}$. There are other orthogonality relations I would generally like to highlight. First noting that $\theta_i \in \text{span}\{\theta_0, u_j,\; j<i\}$, then by use of (\ref{equ: ss Obs}) and the definition of $e_i$ one can show the following.
\begin{subequations}
	\label{equ: inner p}
\begin{align}
\langle \theta_i, u_j \rangle = 0 &\and \langle \theta_i, v_j \rangle = 0 \qquad\;\;\;\text{for}\qquad i\le j;\\
\label{equ: inner p b}\langle X_i, u_j \rangle = 0 &\and \langle X_i, v_j \rangle = 0 \qquad\;\;\text{for}\qquad i < j;\\
\langle X_i, u_j \rangle = S^*_i &\and \langle X_i, v_j \rangle = R_i \qquad\text{for}\qquad i = j;\\ 
\langle e_i, u_j \rangle = 0 &\and \langle e_i, v_j \rangle = 0 \qquad\;\;\;\text{for}\qquad i < j;\\
\langle e_i, u_j \rangle = S^*_i &\and \langle e_i, v_j \rangle = R_i \qquad\;\text{for}\qquad i = j.
\end{align}
\end{subequations}

\subsection{Kalman filter recursions for the innovations}

Kalman developed a recursion to cheaply find the innovations process associated with the observations, which we will derive here.
I like the way Kialath put it, ``It turns out that the recursive construction of the innovations combines nicely with the recursive evolution of the state variables to give a recursion for the innovations...''\cite[p.~312]{kailath2000}.

We seek a recursion for $\{e_i\}$ but start with finding a recursion for $\hat{\theta}_{i|i-1}$ using (\ref{equ: projection}),
\begin{align*}
\hat{\theta}_{i+1|i} &= \sum_{j=0}^i \langle \theta_{i+1}, e_j \rangle R^{-1}_{e,j} e_j\\
&= \left(\sum_{j=0}^{i-1} \langle \theta_{i+1}, e_j \rangle R^{-1}_{e,j} e_j\right) + \langle \theta_{i+1}, e_i \rangle R^{-1}_{e,i} e_i\\
& = \hat{\theta}_{i+1|i-1} + \langle \theta_{i+1}, e_i \rangle R^{-1}_{e,i} e_i
\end{align*}
The first term of the last line undermines the recursion in $\hat{\theta}_{j|j-1}$; however, we can remedy this by conditioning equation (\ref{equ: ss State}) on $\mathcal{E}_{i-1}$ to get
$$\hat{\theta}_{i+1|i-1} = F_i\hat{\theta}_{i|i-1} +G_i\hat{u}_{i|i-1} = F_i\hat{\theta}_{i|i-1} + 0$$
by (\ref{equ: inner p b}). This provides
$$\hat{\theta}_{i+1|i} = F_i\hat{\theta}_{i|i-1} + \langle \theta_{i+1}, e_i \rangle R^{-1}_{e,i} e_i$$
The next thing to do is to make sure we can compute $e_i$ from the $\hat{\theta}_{i|i-1}$'s. If we condition equation (\ref{equ: ss Obs}) on $\mathcal{E}_{i-1}$ then 
\begin{equation}
\hat{X}_{i|i-1} = H_i\hat{\theta}_{i|i-1} + \hat{v}_{i|i-1} = H_i\hat{\theta}_{i|i-1} + 0\label{equ: X hat}
\end{equation}
also by (\ref{equ: inner p b}). So, that
$$e_i = X_i - \hat{X}_{i|i-1} = X_i - H_i\hat{\theta}_{i|i-1}.$$ 
From this point I will use the abbreviated notation $\hat{\theta}_i = \hat{\theta}_{i|i-1}$.

Putting this together, we get a recursion for the innovations
\begin{subequations}
	\begin{empheq}[left=\empheqlbrace]{align}
		\label{equ: theta hat} \hat{\theta}_{i+1} &= F_i\hat{\theta}_{i} + K_ie_i, \quad \hat{\theta}_0 = 0\\
		\label{equ: innov} e_i &= X_i - H_i \hat{\theta}_{i}.
	\end{empheq}
\end{subequations}
where $K_i = \langle \theta_{i+1}, e_i \rangle R^{-1}_{e,i}$. 

Here then is the recursion we sought, which recursively generates the innovations associated with the process $X$. For our interests, and for many others, it would be useful if the $K_i$'s and the $R_{e,i}$'s could be computed off-line, that is, independent of the observations. As it turns out this is possible and I will demonstrate how this may be accomplished.

Let $\tilde{\theta}_i = \theta_i - \hat{\theta}_{i}$ and $P_{i} = \langle \tilde{\theta}_{i},\tilde{\theta}_{i} \rangle$. It turns out that this quantity is very useful in formulating an off-line recursion. It is easy to see by subtracting (\ref{equ: X hat}) from (\ref{equ: ss Obs}) that $e_i = H_i\tilde{\theta}_{i} - v_i$, and this is enough to put $R_{e,i}$ in terms of $P_i$,
because $\hat{\theta}_{i}$ is orthogonal to $v_i$ by (\ref{equ: inner p b}), since $\hat{\theta}_{i}\in \mathfrak{E}_{i-1}$\footnote{
	This means $\hat{\theta}$ is measurable with respect to $\mathfrak{E}_{i-1}$, which is independent of $v_i$.}
, we get
$$R_{e,i} = H_iP_iH^*_i + R_i.$$
To get $K_i$ in terms of only $P_i$ and $R_{e,i}$, we start with $K_i = \langle \theta_{i+1}, e_i \rangle R_{e,i}^{-1}$ and observe that by (\ref{equ: ss Obs}) $\langle X_{i+1}, e_i \rangle = F_i\langle X_{i}, e_i \rangle  + G_i\langle u_i, e_i \rangle$. Reflecting on both inner products in turn provides
\begin{align*}
\langle X_{i}, e_i \rangle &= \langle X_{i}, \tilde{X}_i \rangle H^*_i + \langle X_{i}, v_i \rangle \\
&= \langle \hat{X}_{i} + \tilde{X}_i, \tilde{X}_i \rangle H^*_i + 0 \\
&= P_i H^*_i
\end{align*}
and $\langle u_{i}, e_i \rangle = S_i$, as we have seen. Substituting, we able to find that
$$K_i = (F_iP_iH_i^* + G_iS_i) R_{e,i}^{-1},$$
as desired.
Now, with $R_{e,i}$ and $K_i$ in place a recursion for $P_i$ can be established by
combining equations (\ref{equ: ss State}) and (\ref{equ: theta hat}) and taking the covariance of both sides, a somewhat long calculation results in the discrete Lyapunov recursion,
\begin{align}
\label{equ: DL} P_{i+1} &= F_iP_{i}F_i^* + G_iQ_iG_i^* - K_iR_{e,i}K^*_i,\qquad i\ge 0
\end{align}
which we initialize by $$P_{0} = \langle \hat{\theta}_{0}, \hat{\theta}_{0} \rangle = \langle \theta_0, \theta_0 \rangle = \Pi_0.$$ 


We have just have just found a recursion for the $K_i$ and $R_{e,i}$, which are necessary in the computation of $e_i$, which is independent of observations. It also provides a recursion for the error covariance of the state estimate (which is useful in its own right).  The discussion so far can be summarized into a single theorem
\begin{theorem}
	\label{thm: innov}
	Consider the standard statespace model
	$$\left\{\begin{array}{ r l}
	\theta_{i+1} &= F_i\theta_i + G_iu_i \\
	X_i &= H_i\theta_i + v_i
	\end{array}\right. $$
	for $i\ge 0$, and satisfying (\ref{equ: cond}). The innovations process of $X$ can be recursively computed using the equations
	\begin{subequations}
		\label{equ: innovations}
		\begin{empheq}[left=\empheqlbrace]{align}
			e_i &= X_i - H_i\hat{\theta}_i, \qquad i \ge 0,\\
			\hat{\theta}_{i+1} &= F_i\hat{\theta}_i + K_{p,i}e_i,\qquad \hat{\theta}_0 = 0,
		\end{empheq}
	\end{subequations}
	where 
	\begin{subequations}
		\label{equ: Kal}
		\begin{align}
			\label{equ: Kal Pi}  P_{i+1} &= F_iP_{i}F_i^* + G_iQ_iG_i^* - K_iR_{e,i}K^*_i,\qquad P_0 = \Pi_0 \\
			\label{equ: Kal Rei} R_{e,i} &= H_iP_iH^*_i + R_i \\
			\label{equ: Kal Ki}  K_i &= (F_iP_iH_i^* + G_iS_i) R_{e,i}^{-1}
		\end{align}
	\end{subequations}
	from (\ref{equ: cond}). Here, $P_i = \langle \tilde{\theta}_i,\tilde{\theta}_i \rangle$ where $\tilde{\theta}_i = \theta_i - \hat{\theta}_i$.
\end{theorem}

Observe that the discrete Lyanov recursion (\ref{equ: DL}) can now be written as a discrete Riccati recursion,
\begin{equation}
	P_{i+1} = F_iP_{i}F_i^* + G_iQ_iG_i^* - (F_iP_iH_i^* + G_iS_i)(H_iP_iH^*_i + R_i)^{-1}(H_iP_iF_i^* + S^*_iG^*_i).
	\label{equ: recatti recursion}
\end{equation}

\subsection{The Stationary Case and the Modeling Filter}
\label{sec: Kalman stationary}

So far in this section, all we have assumed about $X$ is that it is a discrete-time stochastic process that has a finite state-space representation defined by (\ref{equ: ss}). But now it is time to assume conditions native to the problem at hand, which will be delineated shortly. These assumptions are afforded by the stationarity of the processes I work with.  The plan is to realize stationarity and adjust the innovations process so as to produce both stationary innovations and stationary state estimators.

Here are the further assumptions I impose on the process $X$, 
\begin{itemize}
	\item the state-space representation for $X$ is time-invariant, meaning, all the parameters are constant in time. So,
	$$F_i = F,\quad G_i=G, \quad H_i = H,\quad Q_i=Q, \quad S_i=S \quad\text{and}\quad R_i = R\qquad \text{for }i\ge 0$$
	additionally I impose that $F$ be stable (meaning its eigenvalues lie strict in the unit circle) and $R$ be strictly positive-definite, $R>0$, also 
	\item $\theta_i$ is in  steady state, meaning $\Pi_i = \Pi$ for $i\ge 0$ where is $\Pi$ is the positive-semi-definite solution to the Lyaponov equation. 
	\begin{equation}
	\Pi = F\Pi F^* + GQG^*.
	\label{equ: state Lyapunov}
	\end{equation}
	 Since $F$ is stable this has a unique solution. 
\end{itemize}

A consequence of these assumptions in that $X$ is wide-sense stationary, and it's model can be written as 
\begin{subequations}
	\label{equ: ss ss}
	\begin{empheq}[left=\empheqlbrace]{align}
	\label{equ: ss ss State}\theta_{i+1} &= F\theta_i + Gu_i \\
	\label{equ: ss ss Obs}X_i &= H\theta_i + v_i
	\end{empheq}
	where
	\begin{equation}
	\E\begin{pmatrix} \theta_0 \\u_i \\ v_i \end{pmatrix}
	\begin{pmatrix} \theta_0 \\u_j \\ v_j\\1 \end{pmatrix}^*
	=\begin{pmatrix}
	\Pi & 0                & 0              & 0 \\
	0     & Q\delta_{ij}   & S\delta_{ij} & 0 \\
	0     & S^*\delta_{ij} & R\delta_{ij} & 0
	\end{pmatrix}
	\label{equ: cond ss}
	\end{equation}
\end{subequations}
It can be show by direct computation that covariance function of $X$, $R_X(i-j)$, can be written as follows \cite[p.~267]{kailath2000}
\begin{equation}
R_X(i-j) = \begin{cases}
	HF^{i-j-1}N & i>j, \\
	R + H\Pi H^* & i = j, \\
	N^*(F^*)^{j-i-1}H^* & i < j,
\end{cases}
\label{equ: autocov X}
\end{equation}
where $N= F\Pi H^* + GS$.


Under these conditions we now apply Theorem \ref{thm: innov} and find that the innovations sequence associated with $X$ has the following state-space representation 
\begin{subequations}
	\label{equ: innovations stationary}
	\begin{align}
	e_i &= X_i - H\hat{\theta}_i, \qquad i \ge 0,\\
	\hat{\theta}_{i+1} &= F\hat{\theta}_{i} + K_{p,i}e_i,\qquad \hat{\theta}_0 = 0,	
	\end{align}
\end{subequations}
	where now
	\begin{subequations}
	\label{equ: Kal stationary}
	\begin{align}
	\label{equ: Kal stationary Pi}  P_{i+1} &= FP_{i}F^* + GQG^* - K_iR_{e,i}K^*_i,\qquad P_0 = \Pi \\
	\label{equ: Kal stationary Rei} R_{e,i} &= HP_iH^* + R \\
	\label{equ: Kal stationary Ki}  K_i &= (FP_iH^* + GS) R_{e,i}^{-1}
	\end{align}
\end{subequations}


Rearranging (\ref{equ: innovations stationary}) slightly we can write
\begin{subequations}
	\label{equ: model}
	\begin{empheq}[left=\empheqlbrace]{align}
		\hat{\theta}_{i+1} &= F\hat{\theta}_{i} + K_{p,i}e_i,\qquad \hat{\theta}_0 = 0,\\
		X_i &= H\hat{\theta}_i + e_i, \qquad i \ge 0.
	\end{empheq}
\end{subequations}
Which provides an alternative way of modeling the process $X$, this time driven by the innovations process $e$. Said differently (\ref{equ: model}) defines a model $\mathcal{L}$ which maps $e$ to $X$. 

Observe that $\mathcal{L}$, clearly causal, is also causally invertable since by algebra we can write,
\begin{subequations}
	\label{equ: white}
	\begin{empheq}[left=\empheqlbrace]{align}
		\hat{\theta}_{i+1} &= F_p\hat{\theta}_{i} + K_{p,i}X_i,\qquad \hat{\theta}_0 = 0,\\
		e_i &= - H\hat{\theta}_i + X_i, \qquad i \ge 0.
	\end{empheq}
\end{subequations}
This contrasts with the original model which took $(\theta_0, u,v)$ to $X$\footnote{
	One thing it has done was remove the additional input variable, the initial state}
, such is causally but not generally causally invertible. 

Now, what can be said about the output $\mathcal{L}w$, for $w = (w_i,\; i = 0, 1, 2, \dots)$ a noise sequence with $\langle w_i, w_j \rangle = R_{e,i}\delta_{ij}$? Is it a stationary process with the same $z$-spectrum as $X$? The output in general is not stationary, since the model, found under (\ref{equ: model}), represents a time-variant system. This begs the question, does this model converge, in time, to a time-invariant model? It turns out that the system parameters $P_i$, $R_{e,i}$, and $K_i$ all converge as $i\rightarrow \infty$. Which suggests a time-invariant causal system that, when fed in a stationary input, will certainly produce a stationary output. A \textit{modeling filter} for $X$ is causal, linear system which constructs the process $X$ by passing a white-noise process through the filter. This is what we are seeking.  


The convergence of $P_i$ has been well studied \cite[Sec.~14.5]{kailath2000} and there are many conditions one can check to ensure that the limit of the recursion converges to the stabilizing solution of the appropriate Racatti equation necessary to provide the causal and casually invertible modeling filter, namily the discrete algebraic Racatti equation (DARE), given by
\begin{align}
P &= FPF^* + GQG^* - (FPH^* + GS)(HPH^* + R)^{-1}(HPF^* + S^*G^*)\\
 &= FPF^* + GQG^* - KR_eK^*
\label{equ: recatti equation}
\end{align}
Since this theory is more problem dependent I will defer the matter of convergence to when we have a specific problem. 


So, assume that
$$\lim_{i\rightarrow \infty} P_i = P$$
So that $P$ in the unique solution to 
\begin{equation}
	P = FPF^* + GQG^* - KR_eK^*
	\label{equ: racatti}
\end{equation}
where $\lim R_{e,i} = R_e = HPH+R$ and $\lim K_i =  K = (FPH^* + GS)R^{-1}_e$, and that $P$ is such that $F_p = F - KH$ is stable. Let us, then, consider the system $$L: w \mapsto Y $$ where $w = (w_i,\; i \ge -\infty)$ with $ \langle w_i,w_j \rangle = R_e\delta_{ij}$ and $Y = (Y_i,\; i \ge -\infty)$\footnote{Notice the slight-of-hand, we now start our processes in the remote past and have them be bilaterally infinite. This is to obviate the need for the state variable to be initialized as an input in order for them to have a variance that obeys the Lyapunov equation below for each finite time} given by 
\begin{subequations}
	\label{equ: modeling}
	\begin{empheq}[left=\empheqlbrace]{align}
		\eta_{i+1} &= F\eta_{i} + Kw_i,\qquad \eta_{-\infty} = 0\\
		Y_i &= H\eta_i+ w_i, \qquad i \ge -\infty.	
	\end{empheq}
Since $\eta_i$ was initialized in the remote past (independent of $w_i$) we have $\langle \eta_i, w_i \rangle = 0$ for $i> -\infty$ and $\langle \eta_i, \eta_i \rangle = \Sigma$ for $i> -\infty$
where 
\begin{align}
\Sigma =F\Sigma F^* +KR_eK^*.
\end{align}
\end{subequations}


We want to check that the autocovariance of $\{Y_i\}$ is the same as for $\{X_i\}$ which is given in (\ref{equ: autocov X}). This can be checked directly. For $i>0$
\begin{align*}
R_Y[i] &= \langle Y_i, Y_0 \rangle \\
	&= \langle H\eta_i+w_i, H\eta_0 + w_0 \rangle \\
	&= \langle HF\eta_{i-1}+HKw_{i-1} + w_i, H\eta_0 + w_0 \rangle \\
	&= \langle HF^2\eta_{i-2}+HFKw_{i-2} + HKw_{i-1} + w_i, H\eta_0 + w_0 \rangle \\
	&= \langle HF^i\eta_{0}+HF^{i-1}Kw_{0} + \dots + HFKw_{i-2} + HKw_{i-1} + w_i, H\eta_0 + w_0 \rangle \\
	&= \langle HF^i\eta_{0}, H\eta_0\rangle + \langle HF^{i-1}Kw_{0} + \dots + HFKw_{i-2} + HKw_{i-1} + w_i, H\eta_0 \rangle \\
	&\qquad +\langle HF^i\eta_{0}, w_0\rangle + \langle HF^{i-1}Kw_{0} + \dots + HFKw_{i-2} + HKw_{i-1} + w_i, w_0 \rangle\\
	&= HF^i\Sigma H^* + 0 + 0 + HF^{i-1}KR_e\\
	&= HF^i\Sigma H^* + HF^{i-1}(FPH^* + GS)\\
	&= HF^{i-1}\big(F(\Sigma + P)H^* + GS)\big),
\end{align*}
a similar calculation for $i<0$ yields
\begin{align*}
R_Y[i] = \big(F(\Sigma + P)H^* + GS)\big)^*(F^*)^{i-1}H^* = R^*_Y[-i],
\end{align*}
and for $i=0$
\begin{align*}
	 R_Y(0) &= \langle Y_0, Y_0 \rangle \\
	 &= \langle H\eta_0+w_0, H\eta_0 + w_0 \rangle \\
	 &= H\Sigma H^* + R_e \\
	 &= H\Sigma H^* + HPH^* + R\\
	 &= H(\Sigma + P)H^* + R
\end{align*}
Now what is $\Sigma + P$, by there definitions 
\begin{align*}
\Sigma + P &= F\Sigma F^* + KR_eK^* + FPF^* + GQG^* - KR_eK^*\\
&= F(\Sigma + P)F^* + GQG^*.
\end{align*}
Meaning that $\Sigma + P$ solves the same discrete Lyapunov equation that $\Pi$ solved in (\ref{equ: state Lyapunov}). We therefore conclude $$\Pi = \Sigma + P$$ and that 
\begin{equation}
R_Y[i] = \begin{cases}
HF^{i-1}N & i>0, \\
R + H\Pi H^* & i = 0, \\
N^*(F^*)^{-i-1}H^* & i < 0,
\end{cases} \qquad= R_X[i]
\label{equ: autocov Y}
\end{equation}
recalling that  $N= F\Pi H^* + GS$, as in equation (\ref{equ: autocov X}). This demonstrates that the causal LTI system $L$ provides a model filter for produce process with the same autocovariance sequence as $X$ given any white noise sequence with covariance $R_e\delta_{ij}$.  

Furthermore, observe that $L$ (given by (\ref{equ: modeling})) is causal and causally invertible, since, by simple algebra again, $L^{-1}$ can be written as
\begin{subequations}
	\label{equ: whitening}
	\begin{empheq}[left=\empheqlbrace]{align}
	\eta_{i+1} &= F_p\eta_{i} + KY_i,\qquad \eta_{-\infty} = 0,\\
	w_i &= - H\eta_i + Y_i, \qquad i \ge -\infty.	
	\end{empheq}
	with $\langle \eta_0, Y_i \rangle = 0$ for $i\ge 0$ and now $\eta_0$ is chosen such that $\langle \eta_0, \eta_0 \rangle = \Sigma_p$ 
	where 
	\begin{align}
	\Sigma_p =F_p\Sigma_p F_p^* +KR_Y[0]K^*
	\end{align}
\end{subequations}
where $F_p = F - KH$ which is stable since $P$ was a stabilizing solution to the Recatti equation given under (\ref{equ: racatti}). This then a whitening filter since.


I now summarize the above discussion into a theorem
\begin{theorem}
	\label{thm: model}
	Let $X = (X_i;\;i > 0)$ be a stationary process with a time-invariant state-space representation of the form given in (\ref{equ: ss ss}). Then, the LTI system $L$ given by (\ref{equ: modeling}) is causal and causally invertable, and constructs a stationary process with the same autocovariance as $X$ by passing in any white-noise process with variance $R_e$.
\end{theorem} 
	


\section{Spectral Factorization by Kalman Filtering}
\label{sec: Factor}

In this section I explain how the above results from Kalman filtering may be employed in finding an approximate spectral factor of a given power spectrum ($z$-spectrum).

Given a $z$-spectrum $S_X(z)$ of a WSS discrete-time process $X_n\in \C^d$ for $n>-\infty$, by definition, $$S_X(z) = \sum_{n=-\infty}^{\infty} R_X[n]z^{-n},$$
where $$R_X[n] = \cov(X_{k+n},X_{k}) = \E X_{k+n}X_{k}^* = \langle X_{k+n},X_{k}\rangle$$ is the covariance. Now, if the decay of the covariance is sufficiently fast it is reasonable to truncate $S_X(z)$ to a Laurent polynomial
$$\tilde{S}_X(z) = \sum_{n=-m}^{m} R_X[n]z^{-n}.$$
However, this isn't always the best way to approximate the spectrum. It is usual to apply some windowing function to the autocovariances. The Parzen windowing function $\Lambda_m$, given below, for some $m>0$ is an example. 
$$\Lambda_m[n] = \begin{cases}
1-6\left(\frac{n}{m}\right)^2+6\left(\frac{|n|}{m}\right)^3, & |n|\le m/2 \\
2\left(1-\frac{|n|}{m}\right)^3, & m/2 < |n| \le m \\
0, & |n| > m
\end{cases}.$$
The Fourier transform $\widehat{\lambda}_m(\omega)$ of the Parzen windowing function is 
$$\widehat{\Lambda}_m(\omega) = \frac{3}{4}m\left(\frac{\sin \pi \omega m/2}{\pi \omega m/2}\right)^4,\qquad -\infty \le \omega \le \infty$$
Let $R_\text{win}[n] = \Lambda_m[n]R_X[n]$ and 
$$S_\text{win}(z) = \sum_{n=-m}^{m} R_\text{win}[n]z^{-n}.$$
The function $S_\text{win}(z)$ is generally a better approximation of $S_{X}(z)$ than $\tilde{S}_{X}(z).$ 


Now that we have approximated the $z$-spectrum we seek to factor by a Laurent polynomial, we can construct a process $Y = \big(Y_i,\; i = 0,1,2, \dots)$ with a finite, time-invariant, state-space model whose $z$-spectrum is exactly $S_\text{win}(z)$, that is $S_{Y}(z) = S_\text{win}(z)$ \cite[p.~488]{sayed2001}. We then take the state-space model and using Kalman, compute the modeling filter. 

The process $Y$ is constructed using the following finite, time-invariant, state-space model:
\begin{subequations}
	\label{equ: lti ss Y}
	\begin{empheq}[left = \empheqlbrace]{align}
		\theta_{i+1} &= F\theta_i + Gv_i \\
		Y_i &= H\theta_i + u_i
	\end{empheq}
	Where\footnote{
		Recall $d$ is the dimension of the observation vectors and $m$ is the degree of the Laurent polynomial approximation of the given $z$-spectrum}
	\begin{equation}
		\E \begin{pmatrix} \theta_0 \\ v_i \\ u_i \end{pmatrix}\begin{pmatrix}\theta_0 \\ v_j \\ u_j \\ 1 \end{pmatrix}^* = 
		\begin{pmatrix} \Pi & 0 & 0 & 0\\
		0 & R\delta_{ij} & S\delta_{ij} & 0 \\ 
		0 & S^*\delta_{ij} & Q\delta_{ij} & 0\end{pmatrix}.
	\end{equation}
\end{subequations}
The parameters $F$, $H$, and $N$ (think of the $N$ from equation (\ref{equ: autocov X}), are defined explicitly by
$$
F = \begin{pmatrix}
0 &   &        &        &   \\
I & 0 &        &        &   \\
& I & 0      &        &   \\
&   & \ddots & \ddots &   \\
&   &        & I      & 0
\end{pmatrix}\in \C^{md\times md},$$
$$
H = \begin{pmatrix}
0 & \dots  & 0       & I 
\end{pmatrix}\in \C^{d\times md},
$$
and
$$N = \begin{pmatrix} R_\text{win}[m] \\ R_\text{win}[m-1] \\ \vdots \\ R_\text{win}[1] \end{pmatrix}\in \C^{md\times d}.$$ 
The remaining quantities are less important almost auxiliary. With the exception of $\Pi$, they have no bearing on the covariance data of the process (recall equation (\ref{equ: autocov X})). The existence of $\Pi$ is guaranteed by the fact that $F$ is stable. It's value never needs to be determined, as we will see. For completeness if one were interested in actually realizing the process $Y$ fro this statespace form the remaining variables must satisfy the following conditions
$$\Pi = F\Pi F^* + GQG^*$$
$$GS = N - F\Pi H^*$$
$$R = R_\text{win}[0] - H\Pi H^*$$
 
Now, provided of all the above can be satisfied it can be show that $Y$ is stationary and its autocovariance is
\begin{align*}
	C_{Y}[i] &= \begin{cases}
		HF^{i-1}N & i > 0 \\
		H\Pi H^* + R & i = 0 \\
		NF^{*(-i-1)}H^* & i < 0
	\end{cases} \\
	&=\begin{cases}
		0 & i > m \\
		R_\text{win}[i]& m \ge i \ge -m \\
		0 & i < -m
	\end{cases} 
\end{align*}
So, that $S_Y(z) = S_{\text{win}}(z) \approx S_X(z)$ as desired. 

In $Y$, we have a stationary process with a time-invariant state-space representation, given under (\ref{equ: lti ss Y}). This representation induces We now invoke Theorem \ref{thm: model}, and assert that the system $M: w = (w_i,\;i> -\infty) \mapsto Y' = (Y'_i,\; i>-\infty)$ given by
\begin{subequations}
	\label{equ: model Y}
	\begin{empheq}[left=\empheqlbrace]{align}
	\eta_{i+1} &= F\eta_{i} + Kw_i,\qquad \eta_{-\infty} = 0\\
	Y'_i &= H\eta_i+ w_i, \qquad i \ge -\infty,	
	\end{empheq}
\end{subequations}
is causal and causally invertible and constructs a stationary process $Y'$ with the same autocovariance of $Y$ by passing in any white-noise process with variance $R_e$. This system $M$ constitutes a modeling filter for $Y$.
Observe that from System (\ref{equ: model Y}) it can be shown that 
$$Y'_i = \sum_{j=1}^{m} HF^{j-1}Kw_{i-j} + w_i = \sum_{j = 0}^{m} K^{\#}_iw_{i-j}$$
where $$K^{\#}_i = \begin{cases}
K_{m-i+1} & 1\le i \le m\\
I & i = 0\end{cases}$$
The notation views $K \in C^{d\times md}$ as a column vector with matrix entries, $(d\times d)$-blocks. So in words, when $K\in C^{d\times md}$ is viewed as a block column vector with $d\times d$-blocks, $K^\#$ is the block-reverse of $K$ appened on top by the $d\times d$ Identity matrix. 
 
And so defines an $MA(m)$ (model) process. 

\section{An Example}
\label{sec: Example}

\appendix
\section{Orthogonality Principle}
\label{app: Ortho proof}

\section{CKMS}

CKMS, named for Chandrasekhar, Kailath, Morf, and Sidhu who in various ways and times contributed to its developement, is a efficient algorithm for computing the Kalman filter recursion for innovations. We will begin by observing who the Kalman filter may be used to find the spectral factors of a rational power spectrum (and approximate that of a nonrational power spectrum) and then from there give details about the CKMS recurssion.


\bibliographystyle{unsrt}
\bibliography{../Notes}
\end{document}
